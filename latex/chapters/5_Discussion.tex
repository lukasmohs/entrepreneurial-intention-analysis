\chapter{Discussion}\label{chapter:discussion}
This paper finds further evidence for Ajzen's \ac{tpb} and rejects the hypothesis that \ac{ee} increases \ac{ei}. The following section is dedicated to the limitations of our findings as well as their practical and theoretical implications. We also direct the attention to possible future research.

\section{Limitations}
Some limitations, methodological as well as practical, should be taken into account. As one important limitation, we identify our study to be of an ex-post research design. Using a purely cross-sectional dataset makes it difficult to differentiate between correlation and causality. Even though we tried to account for this problem by the selection of the control group to be very similar to the treatment group (study degree, current semester), we are aware that our research question demands a longitudinal or panel design \citep{von2010effects, lorz2013entrepreneurship}. Unfortunately, this was impossible in the context of this study due to time constraints. \citet{zhao2005mediating} used matching in order to control for this problem as it was also recommended by \citet{von2010effects}.
Other factors that might reduce the external validity of our results are that our research is limited to one university (\ac{tum}), one faculty (Informatics Faculty) and one course (\ac{seba}).

Nevertheless, we apply many techniques of good research. \citet{lorz2013entrepreneurship} provided methodological improvements from which we incorporate those that are within the scope of this research project. First, we pursue a theory driven approach, which requires the development of the research question and hypothesis prior to developing the methodology and research design. Second, we thoroughly describe the \ac{ec} in chapter \ref{chapter:methodology} and provide a framework for its classification. This is required to make future research results comparable and reproducible. Finally, we use an advanced statistical model by introducing a mediation model in the context of the \ac{tpb}. However, as our hypothesis 2 proves insignificant, a full mediation analysis renders superfluous.


\section{Implications and Future Research}
Our research has various implications on theory and practice of \ac{ee}, which are outlined in the following.

With our first hypothesis, we contribute to Ajzen's \acl{tpb} and reaffirm it. While our analysis shows that \ac{atb} and \ac{pbc} positively influence \ac{ei}, it provides little evidence for an impact of \ac{sn}. This finding is in line with the studies presented in section \ref{sec:tpb-literature} and demonstrates once again that that Ajzen's \ac{tpb} performs well in the entrepreneurship context and that it forms a solid basis for future research in this field. Studies that have been based on the \ac{tpb} can be revalidated. 

At the same time, this opens up the question if this underlying framework could be extended to further decompose the antecedents. Like \citet{zhao2005mediating} demonstrated in his model, analyzing input factors for entrepreneurial self-efficacy and \ac{ei}, Ajzen's antecedents could be extended or tested against further influences.

Given the rejection of our second hypothesis, which claims to show the positive impact of \ac{ee} on \ac{ei}, we conclude that the exact interaction remains unclear. In section \ref{sec:impact-ee-ei}, we present the colliding positions in literature, which already raised the argument for new methodological approaches \citep{von2010effects,lorz2011impact}. The coexistence of significant positive and negative correlation in literature as well as our findings contributes to this argument.

Furthermore, \citet{solesvik2013entrepreneurial} already named the potential difference in study degrees being responsible for the varying formation of \ac{ei} when analyzing engineering and business students. Since the technical students in our study developed a different \ac{ei} than, for instance, American MBA students, which were analyzed by \citet{zhao2005mediating}, two possible implications can be drawn. On the one hand, the selection bias of entrepreneurial students for the corresponding business degree, which was already named by \citet{von2010effects} and \citet{lorz2011impact}, could lead to this difference in attitude. On the other hand, the course content and structure plays a more sophisticated role in the formation of \ac{ei}. In section \ref{sec:defining-ee}, we provide a first approach to classify future studies in order to make the research findings more comparable.

Our practical implications are dedicated to both: policy makers and future researchers. We outline the mismatch between current research findings that are not uniformly supporting the desired effect of \ac{ee} and the situation in practice, which shows a substantial growth in the number of entrepreneurship programs. Institutions such as \ac{tum}, which incorporate and support forms of \ac{ee}, should further investigate the effect on students and potentially align their programs. Among other scholars, \citet{martin2013examining} underlined the relevance of this research for policy makers to maintain the country's entrepreneurial spirit. Based on the research findings, governments can take appropriate measures to improve the \ac{ee} \citep{solesvik2013entrepreneurial}.

%By choosing a mandatory entrepreneurial prototyping course for technology students, we hoped to raise awareness for a different teaching approach. Formerly, focus at universities and in research was on typical venture creation courses (i.e. business plans, value propositions etc.) for business students. We are aware of the fact that the major goal of the SEBA course is not to raise entrepreneurial intent but to teach web development with entrepreneurship on the side. However, we hoped to show a positive side effect of the class on technology students. They are often less exposed to \ac{ee}, so that \ac{ei} might just need to be ignited. Our negative result also calls for a more diverse research where \ac{ee} with different target groups are analyzed.

%Additionally, we also call for longitudinal studies in the field of \ac{ee}, as it provides a firmer basis for the results. Since our resaearch design did allow only to make statements about correlations between the dependent and independent variables, it would be interesting to extend our research design -> longitudinal study. Firstly, identify positive, significant relationship and, secondly, to identify causality. 

%Analyze reasons for a decrease in EI after participating in EE (e.g. students a aware of risks and sacrifices associated with being an entrepreneur).

%Further refinement of the items testing the variable Social norms is necessary since our reliability test did not suffice the empirical standards. 

%First, there is clearly a need for considering different models of entrepreneurship education for different fields of students and also the working population. Most courses where developed for business students and managers, whereas studies show, that the most viable start-ups are founded by technology students \todo{citation}.