\chapter{Discussion}\label{chapter:discussion}

\section{Limitations}
Despite our sophisticated research approach, there might still be implications that should be considered. Foremost, the research is going to be limited to a cross-sectional study on technical students. Our research question clearly demands a longitudinal study, which unfortunately cannot be conducted in our context. Additionally, focussing on technical students at a certain university might afflict the generalizability of the results. However, we address the former concern with the control group and the latter by choosing mandatory courses that the students were not free to choose. With all of this in mind, we are confident about achieving meaningful results with our research.

\begin{itemize}
\item only self reported measures (entrepreneurial test results or grades could be used to measure learnings)
\item one University, one subject, one type of students
\item selected course is limited
\item no longitudinal study
\item Criteria of studies according to Lorz 2013: -	Clusters studies based on 4 criteria: Variables (dependent and independent), Research design (quantitative/qualitative, theory driven), data collection (time of measurement, sampling procedures), data analysis (reliability and validity procedurs, analytical procedures)
\end{itemize}

\section{Theoretical Implications}
(for example: does it contribute to TPB)
By answering our research questions, we contribute to the understanding of the impact of classes on EI in general. In specific, by choosing a mandatory entrepreneurial prototyping course for technology students, we hope to raise awareness for a different teaching approach. Formerly, focus at universities and in research was on typical venture creation courses (i.e. business plans, value propositions etc.). If our hypothesis holds, mandatory entrepreneurial prototyping classes might ignite or strengthen the intent among students who initially had not or weakly aspired such a career. For universities and other educational establishments with a focus on entrepreneurship, this could change the approach of teaching and fostering startups.

\section{Practical Implications}
(for example: can we suggest improvements for EEP)
suggested literature:
\begin{itemize}
\item Lorz 2011: "From a practical point of view, it provides recommendations on how to setup entrepreneurship education programmes and how to facilitate an environment, in which inspirations are triggered."
\end{itemize}
Additionally, it helps to create awareness about different types of university courses and leads to a more diverse discussion of entrepreneurial education.