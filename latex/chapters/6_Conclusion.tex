\chapter{Conclusion}\label{chapter:Conclusion}
The purpose of this paper was to understand how Ajzen's \ac{tpb} explains the formation of \acp{ei} and the impact of \ac{ee} on the formation of \acp{ei}.


For the first aspect, we summarized the literature and showed that the \ac{tpb} is a useful framework to model the complex human behavior. Since the groundwork of Ajzen's \ac{tpb} was first published in 1984 the model proved to be an extremely good and robust predictor of intentions. Building on this, we showed that the literature so far suggests that the model can be extended to the \ac{ei} research. We made contributions to this field of research by using Ajzen's framework in a novel setting and provided evidence that partially reconfirms the \ac{tpb}, since two out of three antecedents are significant predictors of \ac{ei}. 

By researching the second aspect, our goal was twofold. First, we investigated whether there is a correlation of \ac{ee} and \ac{ei}. Second, we wanted to understand the underlying mechanism by which such an effect would be conveyed. Based on the literature, we hypothesized that \ac{ee} positively enhances \ac{ei}. Furthermore, we hypothesized that \ac{tpb} can be used in a mediation analysis with the model's antecedents as the mediators. Neither of these aspects were supported by our data. We were able to relate the identified insignificance to explanations in literature. Missing considerations of the context as well as an analysis of the perception of student's skills may lead to further research.

Another important direction for further research is the understanding of how entrepreneurship is learned. By providing empirical evidence for this aspect the research community contributes to develop best practices in \ac{ee} and thereby to enable policy makers to encourage entrepreneurship on a large scale. As \citet{kuratko2005emergence} has pointed out, this ultimately contributes to the economic and societal development of a country.

