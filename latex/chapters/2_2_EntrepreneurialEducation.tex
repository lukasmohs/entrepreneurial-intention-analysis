\section{\acl{ee}}
\label{sec:ee}

Based on the well-grounded understanding of how intentions develop from a behavioral psychology perspective and its application in the field of entrepreneurship, researchers have gone a step further. \citet{fayolle2015impact} emphasized the need for further research investigating the impact of \acf{ee} on \acp{ei}. \citet{gorman1997some} found in their extensive literature review that \ac{ee} encourages entrepreneurship. Also \citet{kuratko2005emergence} highlighted that it is generally accepted that entrepreneurs are made and not born. However, he claimed that the body of research still lacks clear evidence for a positive impact of \ac{ee} on \acp{ei}. The goal of this section is twofold: First, we develop a clear understanding of what \ac{ee} is and outline three dimensions along which \acp{ec} can be described. Second, we present the findings from the literature of the impact of \ac{ee} on \acp{ei} and derive our hypothesis.

\subsection{Defining \acl{ee}}
\label{sec:defining-ee}
\citet[p. 702]{fayolle2006assessing} defined \acp{ec} "as any pedagogical programme or process of education for entrepreneurial attitudes and skills, which involves developing certain personal qualities". These include an innovative approach to problem solving, the ability to adapt to change, become self-reliant and to develop one's creativity \citep{henry2005entrepreneurship}. According to \citet[p.585]{kuratko2005emergence} and \citet{ronstadt1990the} "students must be prepared to thrive in the 'unstructured and uncertain nature of entrepreneurial environments'". Over the recent years, many different forms of \ac{ee} have been established \citep{kuratko2005emergence}. But \citet{pittaway2007entrepreneurship} pointed out that no clear understanding of \ac{ee}  exists. Based on the works of \citet{pittaway2007entrepreneurship,henry2005entrepreneurship,kuratko2005emergence,garavan1994entrepreneurship}, we identify three dimensions along which the \acp{ec} can be described. These dimensions are the \ac{ec}'s goal and target audience, the content and the pedagogical method, which are outlined in the following.

It is widely accepted that different skills are required during the entrepreneurial process \citep{henry2005entrepreneurship}. Hence, \acp{ec} can be categorized with respect to the stage of the entrepreneurial process, which they address. \citet{jamieson1984education} proposed three "stages" along that process: education about enterprise, education for enterprise and education in enterprise. In the first stage, \ac{ee} focuses on students without a clear intention to start a company. These classes mostly take a theoretical perspective on how to found a company. The second stage of \acp{ec} addresses students who strive to be self-employed. These courses are commonly practice-oriented with the goal to have a company founded or at least a business plan finished by the end of the class. The final stage deals with topics relevant for successful entrepreneurs.

The content of \ac{ee} can be highly diverse ranging from practical tools to devise a business plan to macroeconomic considerations and ethical aspects of entrepreneurship \citep{kuratko2005emergence}. \citet{garavan1994entrepreneurship} pointed out that the content of \acp{ec} often depends on the instructor's personal understanding of entrepreneurship. However, they highlight that common topics include idea generation, business planning and formation, market research and product development. Based on the work of \citet{ronstadt1990the}, \citet{kuratko2005emergence} identified the following fields in \ac{ee}: entrepreneurial vs. managerial domains, venture financing, corporate entrepreneurship, entrepreneurial strategies, psychological aspects predicting future success, risk of an entrepreneurial career, minority entrepreneurs, entrepreneurial spirit, economic and social contributions as well as ethics.

\citet{garavan1994entrepreneurship} and \citet{randolph1979designing} provided a framework to categorize the pedagogical techniques applied in \ac{ee} (see: appendix \ref{fig:gravan}). According to the authors, these can be reflective-theoretical, reflective-applied, active-theoretical, and active-applied. The first pedagogical technique aims at changing the student's knowledge and includes, for instance, theory lectures and required readings. Reflective-applied methods can be psychological reflections, limited discussions and role plays. The third technique aims at changing the understanding of entrepreneurship through focused learning groups, experiments, workshops and coaching. Finally, within the active applied methods, skills and attitudes are changed and methods such as field projects are used. 

\subsection{The Impact of \acl{ee} on \aclp{ei}}
\label{sec:impact-ee-ei}

Building on the definition of \acf{ee}, we now move to the relation of \ac{ee} and \ac{ei}. Within this section, we direct the focus towards the state-of-the-art literature findings in the field of \ac{ee} in order to outline the niche where our study contributes to. 

The underlying question, which many studies tried to answer is whether the participation of an individual in an \ac{ec} leverages their \ac{ei}.
In this context, two coexisting, opposing positions can currently be observed in literature. 
While some studies presented a positive effect through the exposure of  \ac{ee} on \acp{ei} \citep{peterman2003enterprise,zhao2005mediating,solesvik2013entrepreneurial}, others found evidence for a negative effect of participating in \ac{ee} on the \ac{ei} \citep{von2010effects,oosterbeek2010impact}. A selection of remarkable studies in these opposing research positions will be presented in the following. These papers have been selected based on their methodological rigor, which implies a longitudinal pre-test/post-test study design and a significant sample size (see: chapter \ref{chapter:methodology}).

\citet{peterman2003enterprise} found a positive relationship of participating in \ac{ee} on \ac{ei}. The authors applied Shapero's model and analyzed the change of perceived desirability and perceived feasibility of pupils. The study subjects were enrolled in the \acf{yaa} enterprise program over five months and were taught about life cycle of company including marketing, human resources, finance and product development \citep{peterman2003enterprise}. A pre-test/post-test design including control groups was used. One hundred twelve individuals of the \ac{yaa} program and 112 students from the same class, who did not enroll, responded to the questionnaire. The age range was between 15 and 18 years. Desirability and feasibility perceptions of \ac{yaa} participants were significantly higher than the control group's perception. This resulted in \citet{peterman2003enterprise}'s conclusion that underlined the success of the \ac{yaa} program due to its positive influence on the \ac{ei} of pupils.

Also \citet{zhao2005mediating} measured a positive effect of \ac{ee} on \ac{ei} when analyzing the attitude of MBA students at the beginning and end of their degree at five American universities  \citep{zhao2005mediating}. A total of 265 matched responses were collected, which showed a support for both: the significant effect of the MBA program on \ac{ei} as well as the mediating effect of self-efficacy for \ac{ei}. One limitation was that no control group was used to control for external factors.

\citet{solesvik2013entrepreneurial} tested the \ac{tpb} against a data sample of 321 students from three universities in Nikolaev in the Ukraine, where the students' major was differentiated between business and engineering. Since the engineering students did not participate in an entrepreneurship class, \citet{solesvik2013entrepreneurial} could measure the effect of participation on \ac{ei} as well as on the antecedents of \ac{ei}. A significant positive correlation was found between the major in business as well as parental self-employment on \ac{ei}. Additionally, the mediating effect of Ajzen's antecedents could be validated. This led to the conclusion that the engineering focused Ukrainian economy could highly benefit from entrepreneurial elements in engineering degrees due to the potential raise of human capital \citep{solesvik2013entrepreneurial}.

\citet{von2010effects}, however, identified a negative impact of the effects of \ac{ee} on \ac{ei}. Their data has been collected by the department of Business Administration at \ac{lmu} with 196 matching respondents and no control group. The course "Business Planning" is a mandatory element of the curriculum and involves lectures and integrated exercises. The main learning goals were: planning and managing a startup such as creating a business plan, gathering of knowledge about enterprises, entrepreneurship and the gaining of soft skills \citep{von2010effects}. The data was collected right after the kickoff of the project and before receiving the final grade. Initially, 71.4\% of the 196 students indicated an \ac{ei}. After the completion of the course, the amount reduced significantly to 63.8\%. \citet{von2010effects} suggests that the reason for this difference is that individuals realize that they do not possess the necessary skills to pursue a career as an entrepreneur.

Another major study conducted by \citet{oosterbeek2010impact} challenged Ajzen's \ac{tpb}. They examined the impact of \ac{ee} on students with similar degrees, which are enrolled in a vocational college in the south of the Netherlands. The  college had three campuses, of which only one would offer the so-called "Junior Achievement Young Enterprise student mini-company (SMC) program". This program was described by \citet[p. 444]{oosterbeek2010impact} as "the leading entrepreneurship education program in post-secondary education in the Netherlands". This program makes groups of students engage in a small short-time business from setup to liquidation. The goal is an increase in self-confidence, motivation as well as a change of attitude to be proactive, creative and team focused \citep{oosterbeek2010impact}. The outcome of the analysis was an insignificant effect on the self-assessed entrepreneurial skills and even a negative effect on the \ac{ei}. This better self-perception was used by \citet{oosterbeek2010impact} to explain the decreased \ac{ei} when lacking skills to be an entrepreneur.

Finally, \citet{lorz2013entrepreneurship} claimed that many publications showing positive effects of \ac{ee} on \ac{ei} have "methodological deficiencies" \citep[p. 1]{lorz2011impact} due to missing pre-test measurements, missing control-groups or small sample sizes \citep{lorz2011impact}. The author mentioned the study of \citet{peterman2003enterprise} to be the only valid one showing a significant positive effect of \ac{ee} on \ac{ei}. However, his systematic literature review from 2013 named 93 studies in total out of which 65 present a positive effect, three a negative effect, 20 an insignificant, three a contingent and two an unclear result \citep{lorz2013entrepreneurship}. The author himself analyzed three different treatment groups, which participated in additional qualification courses at University of St. Gallen in 2009 or 2010 or in a certificate course  at mixed institutions in 2010 \citep{lorz2011impact}. Additionally, he analyzed a control group at this university. His survey questions were mostly adapted from \citet{linan2009development}. He found no significance in the "change for attitude toward behaviour, subjective norms and entrepreneurial intention" \citep[p. 73]{lorz2011impact}. The only significant impact could be observed for the \ac{pbc} component. Taking his literature review into account, \citet{lorz2013entrepreneurship} further argued that the profile of students influences the effect of \ac{ee} on \ac{ei}.

\subsection{Hypotheses II and III }
\label{sec:hyp-II-III}
Summarizing the research findings from this section, we identify two opposing theories in the current literature. The significance of these opposing study outcomes provides evidence for influence factors being more complex than just the participation in \ac{ee}. With our research, we want to contribute to resolve these contradictory positions in literature. Therefore, our second hypothesis is:\\

\textbf{Hypothesis 2:} \acl{ee} positively influences \acl{ei}.\\

Since we identify Ajzen's model to be the most suitable and most commonly used one for analyzing the impact of \ac{ee} on the antecedents of \ac{ei} (see: section \ref{sec:eim}), our study relies on this model for the final set of hypothesis. These are:\\

\textbf{Hypothesis 3a:} The individuals \acl{atb} positively mediates the effect of \acl{ee} on the participant's \acl{ei}.

\textbf{Hypothesis 3b:} The individuals \aclp{sn} positively mediate the effect of \acl{ee} on the participant's \acl{ei}.

\textbf{Hypothesis 3c} The individuals \acl{pbc} positively mediates the effect of \acl{ee} on the participant's \acl{ei}.

