\chapter{Introduction}\label{chapter:Introduction}

\renewcommand{\thepage}{\arabic{page}}
\setcounter{page}{1}

\section{Principal Topic}
The research field of  \acp{ie}
started  growing in the past years and gained some attention as a powerful framework to predict and explain behavior performed by entrepreneurs. Li{\~n}{\'a}n stated in 2015 that 409 papers dealt with that topic just between 2004 and 2013 \cite{linan2015systematic}.
Besides researchers, academic institutions and governments on a higher level developed an interest in this topic due to the advent of entrepreneurship programs and the potential of increasing the human capital \cite{fayolle2015impact,martin2013examining}.

The concept of intentions is stemming from the field of psychology, where models are built to solve the difficult task of explaining human behavior \cite{ajzen1991theory}. Entrepreneurship is generally classified as a research discipline of Management \cite{shane2000promise}, which makes EI a joint field of both: Psychology and Management. 


Two coexisting  models laid the theoretical foundation for studies of input factors for IE: Shapero introduced the model of the \ac{see} that claimed desirability and feasibility as well as the propensity to act were the three major impact factors on IE \cite{shapero1982social,shapero1984entrepreneurial}. In 1985, Ajzen developed the \ac{tpb} and presented a corresponding  framework, which postulated that attitude, subjective norms and feasibility were the major influence factors \cite{ajzen1985intentions}.

\section{Research Gap}
In recent years, the models of Ajzen and Shapero were applied in  studies to different contexts to test their validity and to study the effect of different circumstances like the cultural background or family environment on EI \cite{laspita2012intergenerational, linan2009development,hayton2002national,mueller2001culture}. 
Krueger and Carsrud outlined the applicability of TPB to the business context, where entrepreneurship training could be analyzed \cite{krueger1993entrepreneurial}. 
But a main field of the \ac{tpb} was dedicated to the \ac{ei} of students \cite{fayolle2015impact}, which will be covered in section \ref{sec:EE}.

In general, two major findings are presented in literature: On the one hand, the positive effect of the participation in entrepreneurship programs or classes on the \ac{ie} were measured in experiments by \todo{lukas: TODO}.
On the other hand, Lorz stated in 2011, that many of these studies had "significant methodological deficiencies" \cite{lorz2011impact}.
In 2013, Li{\~n}{\'a}n summarized the current research situation and outlined several knowledge gaps in the field of \ac{ie} to redirect the focus of researchers. Besides other topics, the nature and effect of entrepreneurial education was mentioned to be still an important subject to study \cite{linan2015systematic}. 



\section{Research Question}
In particular, we will analyze the impact of the participation in a technical prototype class at university on EI, so that our research questions can be formulated as: "How does Ajzen's model explain the formation of EI?" and "How does Ajzen's model explain the impact of entrepreneurial classes (EC) on the formation of \ac{ie}?"