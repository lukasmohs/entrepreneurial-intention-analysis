\chapter{Introduction}\label{chapter:Introduction}

\renewcommand{\thepage}{\arabic{page}}
\setcounter{page}{1}

\section{Principal Topic}
The research field of \acfp{ei} started  growing in the past years and gained some attention as a powerful framework to predict and explain behavior performed by entrepreneurs. \citet{linan2015systematic} stated that 409 papers dealt with this field just between 2004 and 2013.
Besides researchers, academic institutions and governments developed an interest in this topic over the recent years as they have recognized its potential to contribute to the country's economic and social development. That is why a great variety of entrepreneurship programs exist nowadays \citep{fayolle2015impact,martin2013examining,garavan1994entrepreneurship}.

The concept of intentions is stemming from the field of psychology, where models are built to solve the difficult task of explaining human behavior \citep{ajzen1991theory}. Entrepreneurship is generally classified as a research discipline of Management \citep{shane2000promise}, which makes \ac{ei} a joint field of both: psychology and management. 

Two coexisting  models laid the theoretical foundation for studies of input factors for \acp{ei}: Shapero introduced the model of the \acf{see} that claimed desirability and feasibility as well as the propensity to act to be the three major impact factors on \ac{ei} \citep{shapero1982social,shapero1984entrepreneurial}. In 1985, \citet{ajzen1985intentions} laid the foundation for the \ac{tpb}, which he further developed in his seminal paper "The Theory of Planned Behavior" \citep{ajzen1991theory}. In this paper, he presented a corresponding framework, which postulated that \acf{atb}, \acf{sn} and \acf{pbc} were the major influence factors for intentions.

\section{Research Gap}
In recent years, the models of Ajzen and Shapero were applied in studies to different contexts to test their validity under varying circumstances such as diverse cultural backgrounds or family environments \citep{laspita2012intergenerational, linan2009development,hayton2002national,mueller2001culture}. 
\citet{krueger1993entrepreneurial} outlined the applicability of the \ac{tpb} to the business context, where entrepreneurship training could be analyzed. However, a main research focus of the \ac{tpb} was dedicated to the \acl{ei} of students \citep{fayolle2015impact}. Findings regarding this relationship are covered in detail in section \ref{sec:ee}.

In general, two major positions are presented in the literature. On the one hand, a positive effect of the participation in \ac{ee} on the \ac{ei} was found \citep{zhao2005mediating,peterman2003enterprise}. On the other hand, significant negative effects were identified by studies as well \citep{lorz2011impact,von2010effects,oosterbeek2010impact}.

\citet{von2010effects} declared the effect of \acp{ec} on the willingness to engage as an entrepreneur to be unknown and also \citet{oosterbeek2010impact} mentioned that major variances in programs could lead to different outcomes on the students' intention.
\citet{fayolle2014future} summarized the current research situation and outlined several knowledge gaps in the field of \ac{ei} to redirect the focus of researchers. Besides other topics, the nature and effect of \ac{ee} was mentioned to remain an important subject to study \citep{linan2015systematic}.

\section{Research Questions and Structure}
Based on the suggestions from \citet{fayolle2014future}, we analyze the impact of the participation in \acp{ec} at a university on \ac{ei}. Thus, our two distinct research questions can be formulated as follows: 
\begin{center}
\textbf{Research Question I}: "How does Ajzen's \acl{tpb} explain the formation of \aclp{ei}?" 
\end{center}
\begin{center}
\textbf{Research Question II}:  "How does Ajzen's \acl{tpb} explain the impact of \aclp{ec} on the formation of \aclp{ei}?"
\end{center}


The remainder of this paper is structured as follows. First, we explain the underlying theoretical models of \ac{ei} in section \ref{sec:eim} and analyze the current research situation in the field of \ac{ee} in section \ref{sec:ee}. In particular we develop an approach to characterize \ac{ee} and summarize the findings of the impact of \ac{ee} on \ac{ei}. Based on this, we describe our research design in chapter \ref{chapter:methodology} and finally present and discuss our findings with regard to the state-of-the-art research in chapter \ref{chapter:results} and \ref{chapter:discussion}.